\documentclass[main.tex]{subfiles}

\begin{document}
\section{Syntax}

\subsection{Pronouns}
Pronouns in \name{} are organized by person and number. They also mirror the
case system of nouns. The following table describes them:

\begin{tabular}{| c | c | c | c | c | c | c |}
    \hline
    \multicolumn{2}{|c|}{}                                            & \thead{Nominative} & \thead{Accusative} & \thead{Genitive}  & \thead{Dative}    & \thead{Ablative} \\\hline
    \multirow{3}{*}{\rotatebox[origin=c]{90}{Singular}} & \thead{1st} & \textipa{pAn}      & \textipa{pA}       & \textipa{pA:l}    & \textipa{pA:m}    & \textipa{pAr}    \\\cline{2-7}
                                                        & \thead{2nd} & \textipa{non}      & \textipa{no}       & \textipa{no:l}    & \textipa{no:m}    & \textipa{nor}    \\\cline{2-7}
                                                        & \thead{3rd} & \textipa{xon}      & \textipa{xo}       & \textipa{xo:l}    & \textipa{xo:m}    & \textipa{xor}    \\\hline
    \multirow{3}{*}{\rotatebox[origin=c]{90}{Plural}}   & \thead{1st} & \textipa{stepAn}   & \textipa{stepA}    & \textipa{stepA:l} & \textipa{stepA:m} & \textipa{stepAr} \\\cline{2-7}
                                                        & \thead{2nd} & \textipa{stenon}   & \textipa{steno}    & \textipa{steno:l} & \textipa{steno:m} & \textipa{stenor} \\\cline{2-7}
                                                        & \thead{3rd} & \textipa{steGon}   & \textipa{steGo}    & \textipa{steGo:l} & \textipa{steGo:m} & \textipa{steGor} \\\hline
\end{tabular}

\subsection{Possessive Pronouns}
\begin{tabular}{| c | c | c | c | c | c | c |}
    \hline
    \multicolumn{2}{|c|}{}                                            & \thead{Nominative} & \thead{Accusative}  & \thead{Genitive}   & \thead{Dative}     & \thead{Ablative} \\\hline
    \multirow{3}{*}{\rotatebox[origin=c]{90}{Singular}} & \thead{1st} & \textipa{pAm}      & \textipa{pAt}       & \textipa{pA:ln}    & \textipa{pA:rm}    & \textipa{pAl}    \\\cline{2-7}
                                                        & \thead{2nd} & \textipa{nom}      & \textipa{not}       & \textipa{no:ln}    & \textipa{no:rm}    & \textipa{nol}    \\\cline{2-7}
                                                        & \thead{3rd} & \textipa{xom}      & \textipa{xot}       & \textipa{xo:ln}    & \textipa{xo:rm}    & \textipa{xol}    \\\hline
    \multirow{3}{*}{\rotatebox[origin=c]{90}{Plural}}   & \thead{1st} & \textipa{stepAm}   & \textipa{stepAt}    & \textipa{stepA:ln} & \textipa{stepA:rm} & \textipa{stepAl} \\\cline{2-7}
                                                        & \thead{2nd} & \textipa{stenom}   & \textipa{stenot}    & \textipa{steno:ln} & \textipa{steno:rm} & \textipa{stenol} \\\cline{2-7}
                                                        & \thead{3rd} & \textipa{steGom}   & \textipa{steGot}    & \textipa{steGo:ln} & \textipa{steGo:rm} & \textipa{steGol} \\\hline
\end{tabular}

\subsection{Plurals}
% TODO: examples
There is no morphological plural in \name{}. Instead, each noun can be preceded
by a quantifier. These quantifiers can be works like \textit{many} or
\textit{all}, or numbers like \textit{one}, \textit{three}, etc. When no
quantifier is given, the noun's number is ambigous, or can be determined from
context. If it is desired to mark a noun as plural without providing a specific
number, the quantifier \textipa{sten} (literally meaning \textit{number}) is
used.

\subsection{Sentence Order}
\name{} uses SOV sentence order, although this can be modified (to an extent)
for the purpose of poetry or rhetoric due to the case system.
\end{document}
