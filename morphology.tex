\documentclass[main.tex]{subfiles}

\begin{document}
\subsection{Cases}
\name{} has a fairly productive case system. The base form of a noun will take
suffixes depending on the case. All phonological rules then apply to those
suffixes. Special cases of...cases...will be given in the lexicon.

\begin{tabular}{r l}
    \textsc{nominative} & \textipa{-tAn}  \\
    \textsc{accusative} & \textipa{-}     \\
    \textsc{genitive}   & \textipa{-to:l} \\
    \textsc{dative}     & \textipa{-to:m} \\
    \textsc{ablative}   & \textipa{-tAr}
\end{tabular}
% TODO: examples

\subsection{Gender}
% TODO: examples
Rather than the gender system of \name{} being divided into male and female
(and neuter), gender is instead divided into person, prey, predators, and
inanimate. The person category contains all intelligent beings who may be
organized with or interacted with in a social manner. The prey category
contains all non-intelligent animals which the speakers hunt for food. The
predator category contains all non-intelligent (or thought to be
non-intelligent) animals which the speakers are hunted by. The inanimate
category contains effectively everything else.

It's natural to assume that these grammatical categories are unstable, due to
the fact that animal relationships to speakers may change over time. However,
they are actually quite stable within a single region and generation. Most
categories remain unchanged while animal relationships change. Some animals
which may once have been predators are now prey, but they retain the predator
category. Variation mainly manifests itself regionally (due to the fact that
some regions have different traditional predator-prey relationships than other
regions) and inter-generationally (due to natural linguistic innovation).

Sometimes categories can be bent for rhetorical reasons. For example, a brave
adventurer may refer to a dragon using the prey category rather than the
predator category to show how little they think of the threat.

The nouns themselves do not show any morphological sign of the category they
belong to. However, adjectives do. Category is marked with a vowel change in
the final syllable. No change occurs to the base form for the person category.
For the prey category, the base form's final vowel is raised (Open to Mid, and
Mid to Close). For the predator category, the base form's final vowel is
lengthened. And for the inanimate category, the base form's final vowel is
shortened.

Given names tend to not use the prey gender. % TODO: move this to a section on names

Entries in the lexicon will be marked with (c1) for person, (c2) for prey, (c3)
for predator, and (c4) for inanimate. % TODO: come up with better abbreviations

\subsection{Verb Conjugation}
Verbs in \name{} are conjugated by the person and number of their subject.
Irregular forms will be listed specially in the lexicon. The regular forms can
either take suffix forms along with the subject pronouns, or prefix forms
without the subject pronouns.

\subsubsection{Suffix}
\begin{tabular}{| c | c | c |}
    \hline
                & \thead{Singular} & \thead{Plural}    \\\hline
    \thead{1st} & \textipa{-fAn}   & \textipa{-feo}    \\\hline
    \thead{2nd} & \textipa{-Ty}    & \textipa{-T\ae A} \\\hline
    \thead{3rd} & \textipa{-du}    & \textipa{-diu}    \\\hline
\end{tabular}

\subsubsection{Prefix}
\begin{tabular}{| c | c | c |}
    \hline
                & \thead{Singular} & \thead{Plural}   \\\hline
    \thead{1st} & \textipa{pA-}    & \textipa{stepA-} \\\hline
    \thead{2nd} & \textipa{no-}    & \textipa{steno-} \\\hline
    \thead{3rd} & \textipa{o-}     & \textipa{steGo-} \\\hline
\end{tabular}

\subsection{Augmentative}
Nouns can be augmented by appending the suffix \textipa{/-en/}.

\subsection{Diminutives}
Nouns can be diminuated by reduplicating their final syllable, with the second
vowel lengthened.

\subsection{Material Adjectives}
Material nouns such as \textit{gold} (/\textipa{pirin}/ \textit{pirin}) take
the suffix \textipa{/ur/} to become adjectives.

\subsection{Verbs to Nouns}
Verbs can be turned into nouns by adding the suffix \textit{i} (e.g. \textit{pensul} ``communicate'' $\rightarrow$ \textit{pensuli} ``language'').
\end{document}
