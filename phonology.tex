\documentclass[main.tex]{subfiles}

\newcommand{\feat}[1]{$\left[ \begin{array}{l}\text{#1}\end{array}\right]$}

\begin{document}
\name{}'s phonology is modeled after that of Old English.

\subsection{Consonants}
\begin{tabular}{| c | c | c | c | c | c | c |}
    \hline
                        & \thead{Labial}  & \thead{Dental}  & \thead{Alveolar}   & \thead{Post-\\alveolar}          & \thead{Palatal}    & \thead{Velar}     \\\hline
    \thead{Nasal}       & \textipa{m}     &                 & \textipa{n}        &                                  &                    & \textipa{(N)}     \\\hline
    \thead{Stop}        & \textipa{p b}   &                 & \textipa{t d}      &                                  &                    & \textipa{k g}     \\\hline
    \thead{Affricate}   &                 &                 &                    & \multicolumn{2}{c|}{\textipa{tS dZ}}                  &                   \\\hline
    \thead{Fricative}   & \textipa{f (v)} & \textipa{T (D)} & \textipa{s (z)}    & \textipa{(S Z)}                  &                    & \textipa{x (G)}   \\\hline
    \thead{Approximant} &                 & \multicolumn{3}{c|}{\textipa{(\r*l) l}}                                 & \textipa{(\r*j) j} & \textipa{(\*w) w} \\\hline
    \thead{Trill}       &                 & \multicolumn{3}{c|}{\textipa{(\r*r) r}}                                 &                    &                   \\\hline
\end{tabular}

The fricatives \textipa{/f/}, \textipa{/T/}, \textipa{/s/}, \textipa{[S]},
and \textipa{/x/} become voiced when in between voiced sounds. Affricate
\textipa{/tS/} becomes \textipa{[dZ]} in the same environment, although
\textipa{/dZ/} is a phoneme in its own right.

\textipa{[N]} is an allophone of \textipa{/n/} or \textipa{/m/} before any
velar consonant.

\textipa{/r/} and its voiceless counterpart \textipa{/\r*r/} are trills.

\textipa{/\r*l/}, \textipa{/\r*j/}, \textipa{/\*w/}, and \textipa{/\r*r/}
are allophones of their voiced variants following \textipa{/x/}.

\textipa{/s/} and \textipa{[z]} become \textipa{[S]} and \textipa{[Z]}
before back vowels.

Word-final voiced stops become their nasal counterparts. i.e. \textipa{/b/}
becomes \textipa{[m]}, \textipa{/d/} becomes \textipa{[n]}, and
\textipa{/g/} becomes \textipa{[N]}.

\subsection{Vowels}
\begin{tabular}{| c | c | c | c | c |}
    \hline
                    & \multicolumn{2}{c|}{\thead{Front}}             & \multicolumn{2}{c|}{\thead{Back}}\\\hline
                    & \thead{unrounded}       & \thead{rounded}      & \thead{unrounded} & \thead{rounded}\\\hline
    \thead{Close} & \textipa{i i:}          & \textipa{y y:}       &                   & \textipa{u u:}\\\hline
    \thead{Mid}   & \textipa{e e:}          & \textipa{(\o{} \o:)} &                   & \textipa{o o:}\\\hline
    \thead{Open}  & \textipa{\ae{} \ae:}    &                      & \textipa{A A:}    & \\\hline
\end{tabular}

\begin{tabular}{| c | c | c |}
    \hline
                    & \thead{Short}   & \thead{Long}    \\\hline
    \thead{High} & \textipa{iu}    & \textipa{i:u}   \\\hline
    \thead{Mid}  & \textipa{eo}    & \textipa{e:o}   \\\hline
    \thead{Low}  & \textipa{\ae A} & \textipa{\ae:A} \\\hline
\end{tabular}

Each vowel and diphthong in \name{} comes in a pair distinguished by length.

\subsection{Phonotactics}
All syllables in \name{} follow the structure:
\begin{quote}
    (C)(C)V(C)(C)
\end{quote}
Initial consonant clusters can be either \textipa{/st/}, or
\feat{+fricative}\feat{+sonorant}, where \textipa{/r/} counts as a
sonorant.

Final consonant clusters can be either \textipa{/st/} or \textipa{/ln/}.
\end{document}
